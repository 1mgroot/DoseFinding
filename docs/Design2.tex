% !TEX program = pdflatex
\documentclass[11pt]{article}

\usepackage[margin=1in]{geometry}
\usepackage{amsmath, amssymb, bm}
\usepackage{booktabs}
\usepackage{array}

\title{Dose Finding (Part II) Notes}
\author{}
\date{}

\begin{document}
\maketitle

\section{Risk--Benefit tradeoff / Utility}

\subsection{Example (score table)}

Let $Y_I\in\{0,1\}$ (immune response), $Y_E\in\{0,1\}$ (efficacy), $Y_T\in\{0,1\}$ (toxicity).
The score is $w(Y_T,Y_E,Y_I)$ as in the table below.

\paragraph{$Y_I=0$:}
\begin{center}
\begin{tabular}{@{}c*{2}{>{\centering\arraybackslash}p{3.2cm}}@{}}
\toprule
 & $Y_T=0$ & $Y_T=1$ \\
\midrule
$Y_E=0$ & $0$  & $0$  \\
$Y_E=1$ & $80$ & $30$ \\
\bottomrule
\end{tabular}
\end{center}

\paragraph{$Y_I=1$:}
\begin{center}
\begin{tabular}{@{}c*{2}{>{\centering\arraybackslash}p{3.2cm}}@{}}
\toprule
 & $Y_T=0$ & $Y_T=1$ \\
\midrule
$Y_E=0$ & $10$  & $0$  \\
$Y_E=1$ & $100$ & $40$ \\
\bottomrule
\end{tabular}
\end{center}

\subsection{Utility}

\begin{align}
U(d_j)
&=\sum_{y_T=0}^1\sum_{y_E=0}^1\sum_{y_I=0}^1
w(Y_T=y_T,Y_E=y_E,Y_I=y_I)\,
Pr(Y_T=y_T,Y_E=y_E,Y_I=y_I\mid d_j).
\label{eq:U}
\end{align}

A convenient decomposition (as written in the notes) is
\begin{align}
Pr(Y_T,Y_E,Y_I\mid d_j)
&=Pr(Y_T\mid Y_I,d_j)\cdot Pr(Y_E\mid Y_I,d_j)\cdot Pr(Y_I\mid d_j).
\label{eq:decomp}
\end{align}

\subsection{Admissible set (marginal probability)}

\paragraph{(1) No control arm: testing increasing dosages of immunotherapy $d_1,\ldots,d_J$.}
Admissible set using marginal probability:
\begin{align}
Safety: Pr(\pi_{Tj}<\phi_T\mid D_n) &> C_T, \label{eq:adm_noc_T}\\
Activity: Pr(\pi_{Ij}>\phi_I\mid D_n) &> C_I, \label{eq:adm_noc_I}\\
Efficacy: Pr(\pi_{Ej}>\phi_E\mid D_n) &> C_E. \label{eq:adm_noc_E}
\end{align}

\paragraph{(2) With control arm: $d_1,\ldots,d_J$ vs $d_0$.}
Admissible set using marginal probability:
\begin{align}
Safety: Pr\!\Big(\pi_{Tj}<\min(\pi_{T0}+\bar{\pi}_T,\ \phi_T)\mid D_n\Big) &> C_T,
\quad j=1,\ldots,J, \label{eq:adm_ctl_T}\\
Activity: Pr(\pi_{Ij}>\phi_I\mid D_n) &> C_I,
\quad j=1,\ldots,J, \label{eq:adm_ctl_I}\\
Efficacy: Pr\!\Big(\pi_{Ej}>\max(\pi_{E0}+\bar{\pi}_E,\ \phi_E)\mid D_n\Big) &> C_E,
\quad j=1,\ldots,J. \label{eq:adm_ctl_E}
\end{align}

\section{Trial design (Group sequential)}

Trial design: Group sequential approach, consists of ``$S$'' stages.
Let $C_1,\ldots,C_S$ be the pre-specified sample size of the ``$S$'' stages.

\begin{enumerate}
  \item In the first stage, equally randomize $C_1$ patients to $J$ (or $J+1$) arms.
  \item For $s=2,\ldots,S$ stages, based on interim data, update the admissible set $A$
        (safety, activity, and efficacy), and adaptively randomize $C_s$ patients to $d_j\in A$.
        If $A$ is empty, early terminate the trial; no OD selected.
  \item If the trial completes without early stopping, construct the PoC-eligible set:
  \begin{align}
    \mathcal{P}
    = \left\{ j\in A:\ Pr(\pi_{I1}<\delta\,\pi_{Ij}\mid D_n) > C_{\text{PoC}} \right\}.
    \label{eq:PoCset}
  \end{align}
  If $\mathcal{P}=\emptyset$, conclude no PoC and no OD is selected. Otherwise, select the final OD:
  \begin{align}
    OD=\arg\max_{j\in \mathcal{P}}\ \hat{U}(d_j).
    \label{eq:OD}
  \end{align}
\end{enumerate}


\subsection*{Notes}

\begin{enumerate}
  \item Adaptive randomization: assign $C_s$ patients to $A$ based on posterior probabilities
  \begin{align}
    \gamma_j = Pr(OD=d_j\mid D_n),\quad j\in A,
  \end{align}

  \begin{align}
    \frac{\hat{U}(d_j)}{\sum_{j\in A}\hat{U}(d_j)}.
  \end{align}

  \item Ensure reasonable power to find the OD when there is a control:
  assign sufficient but not excessive number of patients to the control, i.e. assign patients to the control based on
  \begin{align}
    r_0 \propto \min(\gamma_{\max},\ 1/|A|),
  \end{align}
  where
  \begin{align}
    \gamma_{\max}=\max\{\gamma_j,\ j\in A\}.
  \end{align}

  \item calibrated $C_{\text{PoC}}$: flat immune response curve, toxicity curve, efficacy curve.
  Control familywise Type I rate is $0.05$, and sample size $\ldots$;
  $80\%$--$90\%$ power, or lower.
\end{enumerate}

\end{document}
